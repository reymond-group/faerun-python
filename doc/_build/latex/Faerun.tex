%% Generated by Sphinx.
\def\sphinxdocclass{report}
\documentclass[letterpaper,10pt,english]{sphinxmanual}
\ifdefined\pdfpxdimen
   \let\sphinxpxdimen\pdfpxdimen\else\newdimen\sphinxpxdimen
\fi \sphinxpxdimen=.75bp\relax

\PassOptionsToPackage{warn}{textcomp}
\usepackage[utf8]{inputenc}
\ifdefined\DeclareUnicodeCharacter
% support both utf8 and utf8x syntaxes
\edef\sphinxdqmaybe{\ifdefined\DeclareUnicodeCharacterAsOptional\string"\fi}
  \DeclareUnicodeCharacter{\sphinxdqmaybe00A0}{\nobreakspace}
  \DeclareUnicodeCharacter{\sphinxdqmaybe2500}{\sphinxunichar{2500}}
  \DeclareUnicodeCharacter{\sphinxdqmaybe2502}{\sphinxunichar{2502}}
  \DeclareUnicodeCharacter{\sphinxdqmaybe2514}{\sphinxunichar{2514}}
  \DeclareUnicodeCharacter{\sphinxdqmaybe251C}{\sphinxunichar{251C}}
  \DeclareUnicodeCharacter{\sphinxdqmaybe2572}{\textbackslash}
\fi
\usepackage{cmap}
\usepackage[T1]{fontenc}
\usepackage{amsmath,amssymb,amstext}
\usepackage{babel}
\usepackage{times}
\usepackage[Bjarne]{fncychap}
\usepackage{sphinx}

\fvset{fontsize=\small}
\usepackage{geometry}

% Include hyperref last.
\usepackage{hyperref}
% Fix anchor placement for figures with captions.
\usepackage{hypcap}% it must be loaded after hyperref.
% Set up styles of URL: it should be placed after hyperref.
\urlstyle{same}

\addto\captionsenglish{\renewcommand{\figurename}{Fig.}}
\addto\captionsenglish{\renewcommand{\tablename}{Table}}
\addto\captionsenglish{\renewcommand{\literalblockname}{Listing}}

\addto\captionsenglish{\renewcommand{\literalblockcontinuedname}{continued from previous page}}
\addto\captionsenglish{\renewcommand{\literalblockcontinuesname}{continues on next page}}
\addto\captionsenglish{\renewcommand{\sphinxnonalphabeticalgroupname}{Non-alphabetical}}
\addto\captionsenglish{\renewcommand{\sphinxsymbolsname}{Symbols}}
\addto\captionsenglish{\renewcommand{\sphinxnumbersname}{Numbers}}

\addto\extrasenglish{\def\pageautorefname{page}}





\title{Faerun}
\date{Jun 21, 2019}
\release{1.0}
\author{}
\newcommand{\sphinxlogo}{\vbox{}}
\renewcommand{\releasename}{Release}
\makeindex
\begin{document}

\pagestyle{empty}
\sphinxmaketitle
\pagestyle{plain}
\sphinxtableofcontents
\pagestyle{normal}
\phantomsection\label{\detokenize{index::doc}}


Here are some teasers.

\noindent\sphinxincludegraphics{{drugbank_2d_light}.png}

\noindent\sphinxincludegraphics{{drugbank_2d_dark}.png}


\chapter{Table of Contents}
\label{\detokenize{index:table-of-contents}}

\section{Getting Started}
\label{\detokenize{tutorial:getting-started}}\label{\detokenize{tutorial::doc}}

\subsection{Installation}
\label{\detokenize{tutorial:installation}}
Faerun can be installed using pip.

\begin{sphinxVerbatim}[commandchars=\\\{\}]
pip install faerun
\end{sphinxVerbatim}

In order to use it in a script, the class \sphinxcode{\sphinxupquote{Faerun}} has to be imported from the package.

\begin{sphinxVerbatim}[commandchars=\\\{\}]
\PYG{k+kn}{from} \PYG{n+nn}{faerun} \PYG{k+kn}{import} \PYG{n}{Faerun}
\end{sphinxVerbatim}

That’s it for the installation.


\subsection{Create a Plot Document}
\label{\detokenize{tutorial:create-a-plot-document}}
In order to plot, a plot document has to be created. To do so, create an instance of the class \sphinxcode{\sphinxupquote{Faerun}}

\begin{sphinxVerbatim}[commandchars=\\\{\}]
\PYG{k+kn}{from} \PYG{n+nn}{faerun} \PYG{k+kn}{import} \PYG{n}{Faerun}

\PYG{n}{f} \PYG{o}{=} \PYG{n}{Faerun}\PYG{p}{(}\PYG{n}{title}\PYG{o}{=}\PYG{l+s+s1}{\PYGZsq{}}\PYG{l+s+s1}{faerun\PYGZhy{}example}\PYG{l+s+s1}{\PYGZsq{}}\PYG{p}{,} \PYG{n}{clear\PYGZus{}color}\PYG{o}{=}\PYG{l+s+s1}{\PYGZsq{}}\PYG{l+s+s1}{\PYGZsh{}222222}\PYG{l+s+s1}{\PYGZsq{}}\PYG{p}{,} \PYG{n}{coords}\PYG{o}{=}\PYG{n+nb+bp}{False}\PYG{p}{,} \PYG{n}{view}\PYG{o}{=}\PYG{l+s+s1}{\PYGZsq{}}\PYG{l+s+s1}{free}\PYG{l+s+s1}{\PYGZsq{}}\PYG{p}{)}
\end{sphinxVerbatim}

Here, we set the \sphinxcode{\sphinxupquote{title}} of the plot document. This will be used as the title of the HTML document. The clear color of the canvas, which is the background color of the plot, is set to \sphinxcode{\sphinxupquote{'\#222222'}} (which is the hex-code for a dark gray).
The drawing of coordinate axes is disabled by setting \sphinxcode{\sphinxupquote{coords=False}} and since we want to draw 3D data, the argument \sphinxcode{\sphinxupquote{view}} is set to \sphinxcode{\sphinxupquote{'free'}} to enable the user to pan and rotate the plot.


\subsection{Preparing Data for Plotting}
\label{\detokenize{tutorial:preparing-data-for-plotting}}
The next step is to prepare the data which is to be plotted. In the tutorial, we will just generate some nice looking data using \sphinxcode{\sphinxupquote{numpy}}.

\begin{sphinxVerbatim}[commandchars=\\\{\}]
\PYG{k+kn}{import} \PYG{n+nn}{numpy} \PYG{k+kn}{as} \PYG{n+nn}{np}

\PYG{n}{x} \PYG{o}{=} \PYG{n}{np}\PYG{o}{.}\PYG{n}{linspace}\PYG{p}{(}\PYG{l+m+mi}{0}\PYG{p}{,} \PYG{l+m+mf}{12.0}\PYG{p}{,} \PYG{l+m+mi}{326}\PYG{p}{)}
\PYG{n}{y} \PYG{o}{=} \PYG{n}{np}\PYG{o}{.}\PYG{n}{sin}\PYG{p}{(}\PYG{n}{np}\PYG{o}{.}\PYG{n}{pi} \PYG{o}{*} \PYG{n}{x}\PYG{p}{)}
\PYG{n}{z} \PYG{o}{=} \PYG{n}{np}\PYG{o}{.}\PYG{n}{cos}\PYG{p}{(}\PYG{n}{np}\PYG{o}{.}\PYG{n}{pi} \PYG{o}{*} \PYG{n}{x}\PYG{p}{)}
\PYG{n}{c} \PYG{o}{=} \PYG{n}{np}\PYG{o}{.}\PYG{n}{random}\PYG{o}{.}\PYG{n}{randint}\PYG{p}{(}\PYG{l+m+mi}{0}\PYG{p}{,} \PYG{l+m+mi}{6}\PYG{p}{,} \PYG{n+nb}{len}\PYG{p}{(}\PYG{n}{x}\PYG{p}{)}\PYG{p}{)}
\end{sphinxVerbatim}

This data can then be wrapped in a \sphinxcode{\sphinxupquote{dict}}. In addition, \sphinxcode{\sphinxupquote{DataFrame}} from the \sphinxcode{\sphinxupquote{pandas}} package are also supported by faerun.
In the example, the same values are used for the colors \sphinxcode{\sphinxupquote{c}} and labels \sphinxcode{\sphinxupquote{labels}}.

\begin{sphinxVerbatim}[commandchars=\\\{\}]
\PYG{n}{data} \PYG{o}{=} \PYG{p}{\PYGZob{}}\PYG{l+s+s1}{\PYGZsq{}}\PYG{l+s+s1}{x}\PYG{l+s+s1}{\PYGZsq{}}\PYG{p}{:} \PYG{n}{x}\PYG{p}{,} \PYG{l+s+s1}{\PYGZsq{}}\PYG{l+s+s1}{y}\PYG{l+s+s1}{\PYGZsq{}}\PYG{p}{:} \PYG{n}{y}\PYG{p}{,} \PYG{l+s+s1}{\PYGZsq{}}\PYG{l+s+s1}{z}\PYG{l+s+s1}{\PYGZsq{}}\PYG{p}{:} \PYG{n}{z}\PYG{p}{,} \PYG{l+s+s1}{\PYGZsq{}}\PYG{l+s+s1}{c}\PYG{l+s+s1}{\PYGZsq{}}\PYG{p}{:} \PYG{n}{c}\PYG{p}{,} \PYG{l+s+s1}{\PYGZsq{}}\PYG{l+s+s1}{labels}\PYG{l+s+s1}{\PYGZsq{}}\PYG{p}{:} \PYG{n}{c}\PYG{p}{\PYGZcb{}}
\end{sphinxVerbatim}


\subsection{Adding a Scatter Layer}
\label{\detokenize{tutorial:adding-a-scatter-layer}}
Given the \sphinxcode{\sphinxupquote{Faerun}} instance and the data, a scatter plot can be created using the method \sphinxcode{\sphinxupquote{add\_scatter}}.

\begin{sphinxVerbatim}[commandchars=\\\{\}]
\PYG{n}{f}\PYG{o}{.}\PYG{n}{add\PYGZus{}scatter}\PYG{p}{(}\PYG{l+s+s1}{\PYGZsq{}}\PYG{l+s+s1}{helix}\PYG{l+s+s1}{\PYGZsq{}}\PYG{p}{,} \PYG{n}{data}\PYG{p}{,} \PYG{n}{shader}\PYG{o}{=}\PYG{l+s+s1}{\PYGZsq{}}\PYG{l+s+s1}{sphere}\PYG{l+s+s1}{\PYGZsq{}}\PYG{p}{,} \PYG{n}{colormap}\PYG{o}{=}\PYG{l+s+s1}{\PYGZsq{}}\PYG{l+s+s1}{Dark2}\PYG{l+s+s1}{\PYGZsq{}}\PYG{p}{,} \PYG{n}{point\PYGZus{}scale}\PYG{o}{=}\PYG{l+m+mf}{5.0}\PYG{p}{,}
              \PYG{n}{categorical}\PYG{o}{=}\PYG{n+nb+bp}{True}\PYG{p}{,} \PYG{n}{has\PYGZus{}legend}\PYG{o}{=}\PYG{n+nb+bp}{True}\PYG{p}{,} \PYG{n}{legend\PYGZus{}labels}\PYG{o}{=}\PYG{p}{[}\PYG{p}{(}\PYG{l+m+mi}{0}\PYG{p}{,} \PYG{l+s+s1}{\PYGZsq{}}\PYG{l+s+s1}{Zero}\PYG{l+s+s1}{\PYGZsq{}}\PYG{p}{)}\PYG{p}{,} \PYG{p}{(}\PYG{l+m+mi}{1}\PYG{p}{,} \PYG{l+s+s1}{\PYGZsq{}}\PYG{l+s+s1}{One}\PYG{l+s+s1}{\PYGZsq{}}\PYG{p}{)}\PYG{p}{]}\PYG{p}{)}
\end{sphinxVerbatim}

The data is added as a scatter layer named helix. The chose shader will render the data points as
spheres (with diffuse and specular lighting) of size 5.0 with colors from the \sphinxcode{\sphinxupquote{matplotlib}} colormap \sphinxcode{\sphinxupquote{'Dark2'}}.
As the \sphinxcode{\sphinxupquote{c}} is categorical, the parameter \sphinxcode{\sphinxupquote{categorical}} is set to \sphinxcode{\sphinxupquote{True}}, otherwise \sphinxcode{\sphinxupquote{matplotlib}} will mess up the values.

Finally, \sphinxcode{\sphinxupquote{has\_legend=True}} adds the scatter layer to the legend and \sphinxcode{\sphinxupquote{legend\_labels}} is a \sphinxcode{\sphinxupquote{list}} of \sphinxcode{\sphinxupquote{tuple}}, associating values with a label.


\subsection{Saving to HTML}
\label{\detokenize{tutorial:saving-to-html}}
The faerun document can the be plotted to an HTML document with an accompanying JavaScript data file.

\begin{sphinxVerbatim}[commandchars=\\\{\}]
\PYG{n}{f}\PYG{o}{.}\PYG{n}{plot}\PYG{p}{(}\PYG{l+s+s1}{\PYGZsq{}}\PYG{l+s+s1}{helix}\PYG{l+s+s1}{\PYGZsq{}}\PYG{p}{)}
\end{sphinxVerbatim}

This saves the plot as \sphinxcode{\sphinxupquote{helix.html{}` and {}`{}`helix.js}}. The files can be opened locally or hosted on any web server.

\noindent\sphinxincludegraphics{{tutorial}.png}


\subsection{Saving to Faerun Data File}
\label{\detokenize{tutorial:saving-to-faerun-data-file}}
The faerun document can also be exported to a faerun data file, which in turn can then be hosted using the \sphinxcode{\sphinxupquote{web}} module.

\begin{sphinxVerbatim}[commandchars=\\\{\}]
\PYG{k+kn}{import} \PYG{n+nn}{pickle}

\PYG{k}{with} \PYG{n+nb}{open}\PYG{p}{(}\PYG{l+s+s1}{\PYGZsq{}}\PYG{l+s+s1}{helix.faerun}\PYG{l+s+s1}{\PYGZsq{}}\PYG{p}{,} \PYG{l+s+s1}{\PYGZsq{}}\PYG{l+s+s1}{wb+}\PYG{l+s+s1}{\PYGZsq{}}\PYG{p}{)} \PYG{k}{as} \PYG{n}{handle}\PYG{p}{:}
    \PYG{n}{pickle}\PYG{o}{.}\PYG{n}{dump}\PYG{p}{(}\PYG{n}{f}\PYG{o}{.}\PYG{n}{create\PYGZus{}python\PYGZus{}data}\PYG{p}{(}\PYG{p}{)}\PYG{p}{,} \PYG{n}{handle}\PYG{p}{,} \PYG{n}{protocol}\PYG{o}{=}\PYG{n}{pickle}\PYG{o}{.}\PYG{n}{HIGHEST\PYGZus{}PROTOCOL}\PYG{p}{)}
\end{sphinxVerbatim}


\subsection{Complete Example}
\label{\detokenize{tutorial:complete-example}}
\begin{sphinxVerbatim}[commandchars=\\\{\}]
\PYG{k+kn}{import} \PYG{n+nn}{pickle}
\PYG{k+kn}{import} \PYG{n+nn}{numpy} \PYG{k+kn}{as} \PYG{n+nn}{np}
\PYG{k+kn}{from} \PYG{n+nn}{faerun} \PYG{k+kn}{import} \PYG{n}{Faerun}

\PYG{k}{def} \PYG{n+nf}{main}\PYG{p}{(}\PYG{p}{)}\PYG{p}{:}
    \PYG{n}{f} \PYG{o}{=} \PYG{n}{Faerun}\PYG{p}{(}\PYG{n}{title}\PYG{o}{=}\PYG{l+s+s1}{\PYGZsq{}}\PYG{l+s+s1}{faerun\PYGZhy{}example}\PYG{l+s+s1}{\PYGZsq{}}\PYG{p}{,} \PYG{n}{clear\PYGZus{}color}\PYG{o}{=}\PYG{l+s+s1}{\PYGZsq{}}\PYG{l+s+s1}{\PYGZsh{}222222}\PYG{l+s+s1}{\PYGZsq{}}\PYG{p}{,} \PYG{n}{coords}\PYG{o}{=}\PYG{n+nb+bp}{False}\PYG{p}{,} \PYG{n}{view}\PYG{o}{=}\PYG{l+s+s1}{\PYGZsq{}}\PYG{l+s+s1}{free}\PYG{l+s+s1}{\PYGZsq{}}\PYG{p}{)}

    \PYG{n}{x} \PYG{o}{=} \PYG{n}{np}\PYG{o}{.}\PYG{n}{linspace}\PYG{p}{(}\PYG{l+m+mi}{0}\PYG{p}{,} \PYG{l+m+mf}{12.0}\PYG{p}{,} \PYG{l+m+mi}{326}\PYG{p}{)}
    \PYG{n}{y} \PYG{o}{=} \PYG{n}{np}\PYG{o}{.}\PYG{n}{sin}\PYG{p}{(}\PYG{n}{np}\PYG{o}{.}\PYG{n}{pi} \PYG{o}{*} \PYG{n}{x}\PYG{p}{)}
    \PYG{n}{z} \PYG{o}{=} \PYG{n}{np}\PYG{o}{.}\PYG{n}{cos}\PYG{p}{(}\PYG{n}{np}\PYG{o}{.}\PYG{n}{pi} \PYG{o}{*} \PYG{n}{x}\PYG{p}{)}
    \PYG{n}{c} \PYG{o}{=} \PYG{n}{np}\PYG{o}{.}\PYG{n}{random}\PYG{o}{.}\PYG{n}{randint}\PYG{p}{(}\PYG{l+m+mi}{0}\PYG{p}{,} \PYG{l+m+mi}{2}\PYG{p}{,} \PYG{n+nb}{len}\PYG{p}{(}\PYG{n}{x}\PYG{p}{)}\PYG{p}{)}

    \PYG{n}{data} \PYG{o}{=} \PYG{p}{\PYGZob{}}\PYG{l+s+s1}{\PYGZsq{}}\PYG{l+s+s1}{x}\PYG{l+s+s1}{\PYGZsq{}}\PYG{p}{:} \PYG{n}{x}\PYG{p}{,} \PYG{l+s+s1}{\PYGZsq{}}\PYG{l+s+s1}{y}\PYG{l+s+s1}{\PYGZsq{}}\PYG{p}{:} \PYG{n}{y}\PYG{p}{,} \PYG{l+s+s1}{\PYGZsq{}}\PYG{l+s+s1}{z}\PYG{l+s+s1}{\PYGZsq{}}\PYG{p}{:} \PYG{n}{z}\PYG{p}{,} \PYG{l+s+s1}{\PYGZsq{}}\PYG{l+s+s1}{c}\PYG{l+s+s1}{\PYGZsq{}}\PYG{p}{:} \PYG{n}{c}\PYG{p}{,} \PYG{l+s+s1}{\PYGZsq{}}\PYG{l+s+s1}{labels}\PYG{l+s+s1}{\PYGZsq{}}\PYG{p}{:} \PYG{n}{c}\PYG{p}{\PYGZcb{}}

    \PYG{n}{f}\PYG{o}{.}\PYG{n}{add\PYGZus{}scatter}\PYG{p}{(}\PYG{l+s+s1}{\PYGZsq{}}\PYG{l+s+s1}{helix}\PYG{l+s+s1}{\PYGZsq{}}\PYG{p}{,} \PYG{n}{data}\PYG{p}{,} \PYG{n}{shader}\PYG{o}{=}\PYG{l+s+s1}{\PYGZsq{}}\PYG{l+s+s1}{sphere}\PYG{l+s+s1}{\PYGZsq{}}\PYG{p}{,} \PYG{n}{colormap}\PYG{o}{=}\PYG{l+s+s1}{\PYGZsq{}}\PYG{l+s+s1}{Dark2}\PYG{l+s+s1}{\PYGZsq{}}\PYG{p}{,} \PYG{n}{point\PYGZus{}scale}\PYG{o}{=}\PYG{l+m+mf}{5.0}\PYG{p}{,}
                \PYG{n}{categorical}\PYG{o}{=}\PYG{n+nb+bp}{True}\PYG{p}{,} \PYG{n}{has\PYGZus{}legend}\PYG{o}{=}\PYG{n+nb+bp}{True}\PYG{p}{,} \PYG{n}{legend\PYGZus{}labels}\PYG{o}{=}\PYG{p}{[}\PYG{p}{(}\PYG{l+m+mi}{0}\PYG{p}{,} \PYG{l+s+s1}{\PYGZsq{}}\PYG{l+s+s1}{Zero}\PYG{l+s+s1}{\PYGZsq{}}\PYG{p}{)}\PYG{p}{,} \PYG{p}{(}\PYG{l+m+mi}{1}\PYG{p}{,} \PYG{l+s+s1}{\PYGZsq{}}\PYG{l+s+s1}{One}\PYG{l+s+s1}{\PYGZsq{}}\PYG{p}{)}\PYG{p}{]}\PYG{p}{)}

    \PYG{n}{f}\PYG{o}{.}\PYG{n}{plot}\PYG{p}{(}\PYG{l+s+s1}{\PYGZsq{}}\PYG{l+s+s1}{helix}\PYG{l+s+s1}{\PYGZsq{}}\PYG{p}{)}

    \PYG{k}{with} \PYG{n+nb}{open}\PYG{p}{(}\PYG{l+s+s1}{\PYGZsq{}}\PYG{l+s+s1}{helix.faerun}\PYG{l+s+s1}{\PYGZsq{}}\PYG{p}{,} \PYG{l+s+s1}{\PYGZsq{}}\PYG{l+s+s1}{wb+}\PYG{l+s+s1}{\PYGZsq{}}\PYG{p}{)} \PYG{k}{as} \PYG{n}{handle}\PYG{p}{:}
        \PYG{n}{pickle}\PYG{o}{.}\PYG{n}{dump}\PYG{p}{(}\PYG{n}{f}\PYG{o}{.}\PYG{n}{create\PYGZus{}python\PYGZus{}data}\PYG{p}{(}\PYG{p}{)}\PYG{p}{,} \PYG{n}{handle}\PYG{p}{,} \PYG{n}{protocol}\PYG{o}{=}\PYG{n}{pickle}\PYG{o}{.}\PYG{n}{HIGHEST\PYGZus{}PROTOCOL}\PYG{p}{)}

\PYG{k}{if} \PYG{n+nv+vm}{\PYGZus{}\PYGZus{}name\PYGZus{}\PYGZus{}} \PYG{o}{==} \PYG{l+s+s1}{\PYGZsq{}}\PYG{l+s+s1}{\PYGZus{}\PYGZus{}main\PYGZus{}\PYGZus{}}\PYG{l+s+s1}{\PYGZsq{}}\PYG{p}{:}
    \PYG{n}{main}\PYG{p}{(}\PYG{p}{)}
\end{sphinxVerbatim}


\section{Web}
\label{\detokenize{host:web}}\label{\detokenize{host::doc}}

\subsection{Introduction}
\label{\detokenize{host:introduction}}
While small plots can easily be created and stored in a single HTML and JavaScript file, which are then loaded completely in the browsers memory, this is not possible for larger data sets due to browser limitations.
In order to solve this problem, Faerun includes a small HTTP server (based on cherrypy) to serve the data to the browser.


\subsection{Creating Faerun Data Files}
\label{\detokenize{host:creating-faerun-data-files}}
As shown in {\hyperref[\detokenize{tutorial::doc}]{\sphinxcrossref{\DUrole{doc}{Getting Started}}}}, Faerun can save data as \sphinxcode{\sphinxupquote{.faerun}} data files using \sphinxcode{\sphinxupquote{pickle}}.

\begin{sphinxVerbatim}[commandchars=\\\{\}]
\PYG{k}{with} \PYG{n+nb}{open}\PYG{p}{(}\PYG{l+s+s1}{\PYGZsq{}}\PYG{l+s+s1}{helix.faerun}\PYG{l+s+s1}{\PYGZsq{}}\PYG{p}{,} \PYG{l+s+s1}{\PYGZsq{}}\PYG{l+s+s1}{wb+}\PYG{l+s+s1}{\PYGZsq{}}\PYG{p}{)} \PYG{k}{as} \PYG{n}{handle}\PYG{p}{:}
    \PYG{n}{pickle}\PYG{o}{.}\PYG{n}{dump}\PYG{p}{(}\PYG{n}{f}\PYG{o}{.}\PYG{n}{create\PYGZus{}python\PYGZus{}data}\PYG{p}{(}\PYG{p}{)}\PYG{p}{,} \PYG{n}{handle}\PYG{p}{,} \PYG{n}{protocol}\PYG{o}{=}\PYG{n}{pickle}\PYG{o}{.}\PYG{n}{HIGHEST\PYGZus{}PROTOCOL}\PYG{p}{)}
\end{sphinxVerbatim}


\subsection{Starting a Faerun Web Server}
\label{\detokenize{host:starting-a-faerun-web-server}}
\begin{sphinxVerbatim}[commandchars=\\\{\}]
\PYG{k+kn}{from} \PYG{n+nn}{faerun} \PYG{k+kn}{import} \PYG{n}{host}

\PYG{n}{host}\PYG{p}{(}\PYG{l+s+s1}{\PYGZsq{}}\PYG{l+s+s1}{helix.faerun}\PYG{l+s+s1}{\PYGZsq{}}\PYG{p}{,} \PYG{n}{label\PYGZus{}type}\PYG{o}{=}\PYG{l+s+s1}{\PYGZsq{}}\PYG{l+s+s1}{default}\PYG{l+s+s1}{\PYGZsq{}}\PYG{p}{,}
     \PYG{n}{theme}\PYG{o}{=}\PYG{l+s+s1}{\PYGZsq{}}\PYG{l+s+s1}{dark}\PYG{l+s+s1}{\PYGZsq{}}\PYG{p}{)}
\end{sphinxVerbatim}


\subsection{Formatting Labels}
\label{\detokenize{host:formatting-labels}}
Labels can be formatted by defining a custom \sphinxcode{\sphinxupquote{label\_formatter}}. If no \sphinxcode{\sphinxupquote{label\_formatter}} is provided to the \sphinxcode{\sphinxupquote{host}} function, the default is used:

\begin{sphinxVerbatim}[commandchars=\\\{\}]
\PYG{n}{label\PYGZus{}formatter} \PYG{o}{=} \PYG{k}{lambda} \PYG{n}{label}\PYG{p}{,} \PYG{n}{index}\PYG{p}{,} \PYG{n}{name}\PYG{p}{:} \PYG{n}{label}\PYG{o}{.}\PYG{n}{split}\PYG{p}{(}\PYG{l+s+s1}{\PYGZsq{}}\PYG{l+s+s1}{\PYGZus{}\PYGZus{}}\PYG{l+s+s1}{\PYGZsq{}}\PYG{p}{)}\PYG{p}{[}\PYG{l+m+mi}{0}\PYG{p}{]}
\end{sphinxVerbatim}

This default splits the label value on \sphinxcode{\sphinxupquote{'\_\_'}} to store different labels and enable search on different values the the displayed labels. See “Searching” for details. Defining a custom label formatter is straight forward. As an example, let’s prefix each label with a string and add their index and layer name:

\begin{sphinxVerbatim}[commandchars=\\\{\}]
\PYG{k}{def} \PYG{n+nf}{custom\PYGZus{}label\PYGZus{}formatter}\PYG{p}{(}\PYG{n}{label}\PYG{p}{,} \PYG{n}{index}\PYG{p}{,} \PYG{n}{name}\PYG{p}{)}\PYG{p}{:}
    \PYG{k}{return} \PYG{n}{f}\PYG{l+s+s1}{\PYGZsq{}}\PYG{l+s+s1}{Example: \PYGZob{}label\PYGZcb{} (\PYGZob{}index\PYGZcb{}, \PYGZob{}name\PYGZcb{})}\PYG{l+s+s1}{\PYGZsq{}}

\PYG{n}{host}\PYG{p}{(}\PYG{l+s+s1}{\PYGZsq{}}\PYG{l+s+s1}{helix.faerun}\PYG{l+s+s1}{\PYGZsq{}}\PYG{p}{,} \PYG{n}{label\PYGZus{}type}\PYG{o}{=}\PYG{l+s+s1}{\PYGZsq{}}\PYG{l+s+s1}{default}\PYG{l+s+s1}{\PYGZsq{}}\PYG{p}{,}
     \PYG{n}{theme}\PYG{o}{=}\PYG{l+s+s1}{\PYGZsq{}}\PYG{l+s+s1}{dark}\PYG{l+s+s1}{\PYGZsq{}}\PYG{p}{,} \PYG{n}{label\PYGZus{}formatter}\PYG{o}{=}\PYG{n}{custom\PYGZus{}label\PYGZus{}formatter}\PYG{p}{)}
\end{sphinxVerbatim}

This function is then called whenever a label is requested from the server. In addition to the argument \sphinxcode{\sphinxupquote{label}}, the arguments \sphinxcode{\sphinxupquote{index}} and \sphinxcode{\sphinxupquote{name}} can be used to further customize the displayed label and represent the integer index of the data point and the data layer they belong to (e.g. the name defined with \sphinxcode{\sphinxupquote{add\_scatter}}).

\noindent\sphinxincludegraphics{{tutorial_host_label}.png}


\subsection{Adding Hyperlinks}
\label{\detokenize{host:adding-hyperlinks}}
Faerun allows to link the data to an arbitrary URL which can be visited upon double-clicking a data point. In order to do this, a \sphinxcode{\sphinxupquote{link\_formatter}} has to be provided. This works similar to customizing the label.

\begin{sphinxVerbatim}[commandchars=\\\{\}]
\PYG{k}{def} \PYG{n+nf}{custom\PYGZus{}link\PYGZus{}formatter}\PYG{p}{(}\PYG{n}{label}\PYG{p}{,} \PYG{n}{index}\PYG{p}{,} \PYG{n}{name}\PYG{p}{)}\PYG{p}{:}
    \PYG{k}{return} \PYG{n}{f}\PYG{l+s+s1}{\PYGZsq{}}\PYG{l+s+s1}{https://www.google.com/search?q=\PYGZob{}label\PYGZcb{}}\PYG{l+s+s1}{\PYGZsq{}}

\PYG{n}{host}\PYG{p}{(}\PYG{l+s+s1}{\PYGZsq{}}\PYG{l+s+s1}{helix.faerun}\PYG{l+s+s1}{\PYGZsq{}}\PYG{p}{,} \PYG{n}{label\PYGZus{}type}\PYG{o}{=}\PYG{l+s+s1}{\PYGZsq{}}\PYG{l+s+s1}{default}\PYG{l+s+s1}{\PYGZsq{}}\PYG{p}{,}
     \PYG{n}{theme}\PYG{o}{=}\PYG{l+s+s1}{\PYGZsq{}}\PYG{l+s+s1}{dark}\PYG{l+s+s1}{\PYGZsq{}}\PYG{p}{,} \PYG{n}{link\PYGZus{}formatter}\PYG{o}{=}\PYG{n}{custom\PYGZus{}link\PYGZus{}formatter}\PYG{p}{)}
\end{sphinxVerbatim}


\subsection{Searching}
\label{\detokenize{host:searching}}
The hosted version of a Faerun visualization also allows for searching. As a default, the search searches for exact matches in labels (substring or regex searches are not possible at this time).

\noindent\sphinxincludegraphics{{tutorial_host_search}.png}

However, the search can be customized. As described in “Formatting Labels”, additional label values can be added by separating them using \sphinxcode{\sphinxupquote{'\_\_'}}.

\begin{sphinxVerbatim}[commandchars=\\\{\}]
\PYG{n}{c} \PYG{o}{=} \PYG{n}{np}\PYG{o}{.}\PYG{n}{random}\PYG{o}{.}\PYG{n}{randint}\PYG{p}{(}\PYG{l+m+mi}{0}\PYG{p}{,} \PYG{l+m+mi}{2}\PYG{p}{,} \PYG{n+nb}{len}\PYG{p}{(}\PYG{n}{x}\PYG{p}{)}\PYG{p}{)}
\PYG{n}{labels} \PYG{o}{=} \PYG{p}{[}\PYG{l+s+s1}{\PYGZsq{}}\PYG{l+s+s1}{\PYGZsq{}}\PYG{p}{]} \PYG{o}{*} \PYG{n+nb}{len}\PYG{p}{(}\PYG{n}{c}\PYG{p}{)}

\PYG{k}{for} \PYG{n}{i}\PYG{p}{,} \PYG{n}{e} \PYG{o+ow}{in} \PYG{n+nb}{enumerate}\PYG{p}{(}\PYG{n}{c}\PYG{p}{)}\PYG{p}{:}
    \PYG{n}{labels}\PYG{p}{[}\PYG{n}{i}\PYG{p}{]} \PYG{o}{=} \PYG{n+nb}{str}\PYG{p}{(}\PYG{n}{e}\PYG{p}{)} \PYG{o}{+} \PYG{l+s+s1}{\PYGZsq{}}\PYG{l+s+s1}{\PYGZus{}\PYGZus{}}\PYG{l+s+s1}{\PYGZsq{}} \PYG{o}{+} \PYG{n+nb}{str}\PYG{p}{(}\PYG{n}{i} \PYG{o}{\PYGZpc{}} \PYG{l+m+mi}{20}\PYG{p}{)}

\PYG{n}{data} \PYG{o}{=} \PYG{p}{\PYGZob{}}\PYG{l+s+s1}{\PYGZsq{}}\PYG{l+s+s1}{x}\PYG{l+s+s1}{\PYGZsq{}}\PYG{p}{:} \PYG{n}{x}\PYG{p}{,} \PYG{l+s+s1}{\PYGZsq{}}\PYG{l+s+s1}{y}\PYG{l+s+s1}{\PYGZsq{}}\PYG{p}{:} \PYG{n}{y}\PYG{p}{,} \PYG{l+s+s1}{\PYGZsq{}}\PYG{l+s+s1}{z}\PYG{l+s+s1}{\PYGZsq{}}\PYG{p}{:} \PYG{n}{z}\PYG{p}{,} \PYG{l+s+s1}{\PYGZsq{}}\PYG{l+s+s1}{c}\PYG{l+s+s1}{\PYGZsq{}}\PYG{p}{:} \PYG{n}{c}\PYG{p}{,} \PYG{l+s+s1}{\PYGZsq{}}\PYG{l+s+s1}{labels}\PYG{l+s+s1}{\PYGZsq{}}\PYG{p}{:} \PYG{n}{labels}\PYG{p}{\PYGZcb{}}
\end{sphinxVerbatim}

The above examples adds an additional label value and as default, the second label value is then used by the search.

\noindent\sphinxincludegraphics{{tutorial_host_search_2}.png}

If there are additional label values, the search index can be set using the \sphinxcode{\sphinxupquote{search\_index}} argument.


\subsection{Add Info / Documentation}
\label{\detokenize{host:add-info-documentation}}
As the visualization is ready to be deployed to a publicly accessible web server, it might be of interest to add a documentation. The \sphinxcode{\sphinxupquote{host}} method supports the argument \sphinxcode{\sphinxupquote{info}} that accepts a (markdown formatted) string. This information is the desplayed on the generated web page.

\begin{sphinxVerbatim}[commandchars=\\\{\}]
\PYG{n}{info} \PYG{o}{=} \PYG{p}{(}\PYG{l+s+s1}{\PYGZsq{}}\PYG{l+s+s1}{\PYGZsh{}Welcome to Fearun}\PYG{l+s+s1}{\PYGZsq{}}\PYG{p}{,}
        \PYG{l+s+s1}{\PYGZsq{}}\PYG{l+s+s1}{This is a small Faerun example.}\PYG{l+s+s1}{\PYGZsq{}}
        \PYG{l+s+s1}{\PYGZsq{}}\PYG{l+s+s1}{\PYGZsq{}}\PYG{p}{,}
        \PYG{l+s+s1}{\PYGZsq{}}\PYG{l+s+s1}{Yay markdown! This means that you can easily:}\PYG{l+s+s1}{\PYGZsq{}}\PYG{p}{,}
        \PYG{l+s+s1}{\PYGZsq{}}\PYG{l+s+s1}{\PYGZhy{} Add lists}\PYG{l+s+s1}{\PYGZsq{}}\PYG{p}{,}
        \PYG{l+s+s1}{\PYGZsq{}}\PYG{l+s+s1}{\PYGZhy{} Build tables}\PYG{l+s+s1}{\PYGZsq{}}\PYG{p}{,}
        \PYG{l+s+s1}{\PYGZsq{}}\PYG{l+s+s1}{\PYGZhy{} Insert images and links}\PYG{l+s+s1}{\PYGZsq{}}\PYG{p}{,}
        \PYG{l+s+s1}{\PYGZsq{}}\PYG{l+s+s1}{\PYGZhy{} Add code examples}\PYG{l+s+s1}{\PYGZsq{}}\PYG{p}{,}
        \PYG{l+s+s1}{\PYGZsq{}}\PYG{l+s+s1}{\PYGZhy{} ...}\PYG{l+s+s1}{\PYGZsq{}}
       \PYG{p}{)}

\PYG{n}{host}\PYG{p}{(}\PYG{l+s+s1}{\PYGZsq{}}\PYG{l+s+s1}{helix.faerun}\PYG{l+s+s1}{\PYGZsq{}}\PYG{p}{,} \PYG{n}{label\PYGZus{}type}\PYG{o}{=}\PYG{l+s+s1}{\PYGZsq{}}\PYG{l+s+s1}{default}\PYG{l+s+s1}{\PYGZsq{}}\PYG{p}{,} \PYG{n}{theme}\PYG{o}{=}\PYG{l+s+s1}{\PYGZsq{}}\PYG{l+s+s1}{dark}\PYG{l+s+s1}{\PYGZsq{}}\PYG{p}{,}
    \PYG{n}{label\PYGZus{}formatter}\PYG{o}{=}\PYG{n}{custom\PYGZus{}label\PYGZus{}formatter}\PYG{p}{,} \PYG{n}{link\PYGZus{}formatter}\PYG{o}{=}\PYG{n}{custom\PYGZus{}link\PYGZus{}formatter}\PYG{p}{,}
    \PYG{n}{info}\PYG{o}{=}\PYG{l+s+s1}{\PYGZsq{}}\PYG{l+s+se}{\PYGZbs{}n}\PYG{l+s+s1}{\PYGZsq{}}\PYG{o}{.}\PYG{n}{join}\PYG{p}{(}\PYG{n}{info}\PYG{p}{)}\PYG{p}{)}
\end{sphinxVerbatim}

An info button is then shown next to the screenshot button, which upon click opens a window containing the info.


\subsection{Complete Example}
\label{\detokenize{host:complete-example}}
\begin{sphinxVerbatim}[commandchars=\\\{\}]
\PYG{k+kn}{import} \PYG{n+nn}{pickle}
\PYG{k+kn}{import} \PYG{n+nn}{numpy} \PYG{k+kn}{as} \PYG{n+nn}{np}
\PYG{k+kn}{from} \PYG{n+nn}{faerun} \PYG{k+kn}{import} \PYG{n}{Faerun}\PYG{p}{,} \PYG{n}{host}


\PYG{k}{def} \PYG{n+nf}{main}\PYG{p}{(}\PYG{p}{)}\PYG{p}{:}
    \PYG{n}{f} \PYG{o}{=} \PYG{n}{Faerun}\PYG{p}{(}\PYG{n}{title}\PYG{o}{=}\PYG{l+s+s1}{\PYGZsq{}}\PYG{l+s+s1}{faerun\PYGZhy{}example}\PYG{l+s+s1}{\PYGZsq{}}\PYG{p}{,} \PYG{n}{clear\PYGZus{}color}\PYG{o}{=}\PYG{l+s+s1}{\PYGZsq{}}\PYG{l+s+s1}{\PYGZsh{}222222}\PYG{l+s+s1}{\PYGZsq{}}\PYG{p}{,} \PYG{n}{coords}\PYG{o}{=}\PYG{n+nb+bp}{False}\PYG{p}{,} \PYG{n}{view}\PYG{o}{=}\PYG{l+s+s1}{\PYGZsq{}}\PYG{l+s+s1}{free}\PYG{l+s+s1}{\PYGZsq{}}\PYG{p}{)}

    \PYG{n}{x} \PYG{o}{=} \PYG{n}{np}\PYG{o}{.}\PYG{n}{linspace}\PYG{p}{(}\PYG{l+m+mi}{0}\PYG{p}{,} \PYG{l+m+mf}{12.0}\PYG{p}{,} \PYG{l+m+mi}{326}\PYG{p}{)}
    \PYG{n}{y} \PYG{o}{=} \PYG{n}{np}\PYG{o}{.}\PYG{n}{sin}\PYG{p}{(}\PYG{n}{np}\PYG{o}{.}\PYG{n}{pi} \PYG{o}{*} \PYG{n}{x}\PYG{p}{)}
    \PYG{n}{z} \PYG{o}{=} \PYG{n}{np}\PYG{o}{.}\PYG{n}{cos}\PYG{p}{(}\PYG{n}{np}\PYG{o}{.}\PYG{n}{pi} \PYG{o}{*} \PYG{n}{x}\PYG{p}{)}
    \PYG{n}{c} \PYG{o}{=} \PYG{n}{np}\PYG{o}{.}\PYG{n}{random}\PYG{o}{.}\PYG{n}{randint}\PYG{p}{(}\PYG{l+m+mi}{0}\PYG{p}{,} \PYG{l+m+mi}{2}\PYG{p}{,} \PYG{n+nb}{len}\PYG{p}{(}\PYG{n}{x}\PYG{p}{)}\PYG{p}{)}

    \PYG{n}{labels} \PYG{o}{=} \PYG{p}{[}\PYG{l+s+s1}{\PYGZsq{}}\PYG{l+s+s1}{\PYGZsq{}}\PYG{p}{]} \PYG{o}{*} \PYG{n+nb}{len}\PYG{p}{(}\PYG{n}{c}\PYG{p}{)}

    \PYG{k}{for} \PYG{n}{i}\PYG{p}{,} \PYG{n}{e} \PYG{o+ow}{in} \PYG{n+nb}{enumerate}\PYG{p}{(}\PYG{n}{c}\PYG{p}{)}\PYG{p}{:}
        \PYG{n}{labels}\PYG{p}{[}\PYG{n}{i}\PYG{p}{]} \PYG{o}{=} \PYG{n+nb}{str}\PYG{p}{(}\PYG{n}{e}\PYG{p}{)} \PYG{o}{+} \PYG{l+s+s1}{\PYGZsq{}}\PYG{l+s+s1}{\PYGZus{}\PYGZus{}}\PYG{l+s+s1}{\PYGZsq{}} \PYG{o}{+} \PYG{n+nb}{str}\PYG{p}{(}\PYG{n}{i} \PYG{o}{\PYGZpc{}} \PYG{l+m+mi}{20}\PYG{p}{)}

    \PYG{n}{data} \PYG{o}{=} \PYG{p}{\PYGZob{}}\PYG{l+s+s1}{\PYGZsq{}}\PYG{l+s+s1}{x}\PYG{l+s+s1}{\PYGZsq{}}\PYG{p}{:} \PYG{n}{x}\PYG{p}{,} \PYG{l+s+s1}{\PYGZsq{}}\PYG{l+s+s1}{y}\PYG{l+s+s1}{\PYGZsq{}}\PYG{p}{:} \PYG{n}{y}\PYG{p}{,} \PYG{l+s+s1}{\PYGZsq{}}\PYG{l+s+s1}{z}\PYG{l+s+s1}{\PYGZsq{}}\PYG{p}{:} \PYG{n}{z}\PYG{p}{,} \PYG{l+s+s1}{\PYGZsq{}}\PYG{l+s+s1}{c}\PYG{l+s+s1}{\PYGZsq{}}\PYG{p}{:} \PYG{n}{c}\PYG{p}{,} \PYG{l+s+s1}{\PYGZsq{}}\PYG{l+s+s1}{labels}\PYG{l+s+s1}{\PYGZsq{}}\PYG{p}{:} \PYG{n}{labels}\PYG{p}{\PYGZcb{}}

    \PYG{n}{f}\PYG{o}{.}\PYG{n}{add\PYGZus{}scatter}\PYG{p}{(}\PYG{l+s+s1}{\PYGZsq{}}\PYG{l+s+s1}{helix}\PYG{l+s+s1}{\PYGZsq{}}\PYG{p}{,} \PYG{n}{data}\PYG{p}{,} \PYG{n}{shader}\PYG{o}{=}\PYG{l+s+s1}{\PYGZsq{}}\PYG{l+s+s1}{sphere}\PYG{l+s+s1}{\PYGZsq{}}\PYG{p}{,} \PYG{n}{colormap}\PYG{o}{=}\PYG{l+s+s1}{\PYGZsq{}}\PYG{l+s+s1}{Dark2}\PYG{l+s+s1}{\PYGZsq{}}\PYG{p}{,} \PYG{n}{point\PYGZus{}scale}\PYG{o}{=}\PYG{l+m+mf}{5.0}\PYG{p}{,}
                \PYG{n}{categorical}\PYG{o}{=}\PYG{n+nb+bp}{True}\PYG{p}{,} \PYG{n}{has\PYGZus{}legend}\PYG{o}{=}\PYG{n+nb+bp}{True}\PYG{p}{,} \PYG{n}{legend\PYGZus{}labels}\PYG{o}{=}\PYG{p}{[}\PYG{p}{(}\PYG{l+m+mi}{0}\PYG{p}{,} \PYG{l+s+s1}{\PYGZsq{}}\PYG{l+s+s1}{Zero}\PYG{l+s+s1}{\PYGZsq{}}\PYG{p}{)}\PYG{p}{,} \PYG{p}{(}\PYG{l+m+mi}{1}\PYG{p}{,} \PYG{l+s+s1}{\PYGZsq{}}\PYG{l+s+s1}{One}\PYG{l+s+s1}{\PYGZsq{}}\PYG{p}{)}\PYG{p}{]}\PYG{p}{)}

    \PYG{n}{f}\PYG{o}{.}\PYG{n}{plot}\PYG{p}{(}\PYG{l+s+s1}{\PYGZsq{}}\PYG{l+s+s1}{helix}\PYG{l+s+s1}{\PYGZsq{}}\PYG{p}{)}

    \PYG{k}{with} \PYG{n+nb}{open}\PYG{p}{(}\PYG{l+s+s1}{\PYGZsq{}}\PYG{l+s+s1}{helix.faerun}\PYG{l+s+s1}{\PYGZsq{}}\PYG{p}{,} \PYG{l+s+s1}{\PYGZsq{}}\PYG{l+s+s1}{wb+}\PYG{l+s+s1}{\PYGZsq{}}\PYG{p}{)} \PYG{k}{as} \PYG{n}{handle}\PYG{p}{:}
        \PYG{n}{pickle}\PYG{o}{.}\PYG{n}{dump}\PYG{p}{(}\PYG{n}{f}\PYG{o}{.}\PYG{n}{create\PYGZus{}python\PYGZus{}data}\PYG{p}{(}\PYG{p}{)}\PYG{p}{,} \PYG{n}{handle}\PYG{p}{,} \PYG{n}{protocol}\PYG{o}{=}\PYG{n}{pickle}\PYG{o}{.}\PYG{n}{HIGHEST\PYGZus{}PROTOCOL}\PYG{p}{)}

    \PYG{k}{def} \PYG{n+nf}{custom\PYGZus{}label\PYGZus{}formatter}\PYG{p}{(}\PYG{n}{label}\PYG{p}{,} \PYG{n}{index}\PYG{p}{,} \PYG{n}{name}\PYG{p}{)}\PYG{p}{:}
        \PYG{k}{return} \PYG{n}{f}\PYG{l+s+s1}{\PYGZsq{}}\PYG{l+s+s1}{Example: \PYGZob{}label\PYGZcb{} (\PYGZob{}index\PYGZcb{}, \PYGZob{}name\PYGZcb{})}\PYG{l+s+s1}{\PYGZsq{}}

    \PYG{k}{def} \PYG{n+nf}{custom\PYGZus{}link\PYGZus{}formatter}\PYG{p}{(}\PYG{n}{label}\PYG{p}{,} \PYG{n}{index}\PYG{p}{,} \PYG{n}{name}\PYG{p}{)}\PYG{p}{:}
        \PYG{k}{return} \PYG{n}{f}\PYG{l+s+s1}{\PYGZsq{}}\PYG{l+s+s1}{https://www.google.com/search?q=\PYGZob{}label\PYGZcb{}}\PYG{l+s+s1}{\PYGZsq{}}

    \PYG{n}{info} \PYG{o}{=} \PYG{p}{(}\PYG{l+s+s1}{\PYGZsq{}}\PYG{l+s+s1}{\PYGZsh{}Welcome to Fearun}\PYG{l+s+s1}{\PYGZsq{}}\PYG{p}{,}
            \PYG{l+s+s1}{\PYGZsq{}}\PYG{l+s+s1}{This is a small Faerun example.}\PYG{l+s+s1}{\PYGZsq{}}
            \PYG{l+s+s1}{\PYGZsq{}}\PYG{l+s+s1}{\PYGZsq{}}\PYG{p}{,}
            \PYG{l+s+s1}{\PYGZsq{}}\PYG{l+s+s1}{Yay markdown! This means that you can easily:}\PYG{l+s+s1}{\PYGZsq{}}\PYG{p}{,}
            \PYG{l+s+s1}{\PYGZsq{}}\PYG{l+s+s1}{\PYGZhy{} Add lists}\PYG{l+s+s1}{\PYGZsq{}}\PYG{p}{,}
            \PYG{l+s+s1}{\PYGZsq{}}\PYG{l+s+s1}{\PYGZhy{} Build tables}\PYG{l+s+s1}{\PYGZsq{}}\PYG{p}{,}
            \PYG{l+s+s1}{\PYGZsq{}}\PYG{l+s+s1}{\PYGZhy{} Insert images and links}\PYG{l+s+s1}{\PYGZsq{}}\PYG{p}{,}
            \PYG{l+s+s1}{\PYGZsq{}}\PYG{l+s+s1}{\PYGZhy{} Add code examples}\PYG{l+s+s1}{\PYGZsq{}}\PYG{p}{,}
            \PYG{l+s+s1}{\PYGZsq{}}\PYG{l+s+s1}{\PYGZhy{} ...}\PYG{l+s+s1}{\PYGZsq{}}
        \PYG{p}{)}

    \PYG{n}{host}\PYG{p}{(}\PYG{l+s+s1}{\PYGZsq{}}\PYG{l+s+s1}{helix.faerun}\PYG{l+s+s1}{\PYGZsq{}}\PYG{p}{,} \PYG{n}{label\PYGZus{}type}\PYG{o}{=}\PYG{l+s+s1}{\PYGZsq{}}\PYG{l+s+s1}{default}\PYG{l+s+s1}{\PYGZsq{}}\PYG{p}{,} \PYG{n}{theme}\PYG{o}{=}\PYG{l+s+s1}{\PYGZsq{}}\PYG{l+s+s1}{dark}\PYG{l+s+s1}{\PYGZsq{}}\PYG{p}{,}
        \PYG{n}{label\PYGZus{}formatter}\PYG{o}{=}\PYG{n}{custom\PYGZus{}label\PYGZus{}formatter}\PYG{p}{,} \PYG{n}{link\PYGZus{}formatter}\PYG{o}{=}\PYG{n}{custom\PYGZus{}link\PYGZus{}formatter}\PYG{p}{,}
        \PYG{n}{info}\PYG{o}{=}\PYG{l+s+s1}{\PYGZsq{}}\PYG{l+s+se}{\PYGZbs{}n}\PYG{l+s+s1}{\PYGZsq{}}\PYG{o}{.}\PYG{n}{join}\PYG{p}{(}\PYG{n}{info}\PYG{p}{)}\PYG{p}{)}


\PYG{k}{if} \PYG{n+nv+vm}{\PYGZus{}\PYGZus{}name\PYGZus{}\PYGZus{}} \PYG{o}{==} \PYG{l+s+s1}{\PYGZsq{}}\PYG{l+s+s1}{\PYGZus{}\PYGZus{}main\PYGZus{}\PYGZus{}}\PYG{l+s+s1}{\PYGZsq{}}\PYG{p}{:}
    \PYG{n}{main}\PYG{p}{(}\PYG{p}{)}
\end{sphinxVerbatim}


\section{Documentation}
\label{\detokenize{documentation:documentation}}\label{\detokenize{documentation::doc}}

\subsection{Faerun}
\label{\detokenize{documentation:faerun}}\index{Faerun (class in faerun)@\spxentry{Faerun}\spxextra{class in faerun}}

\begin{fulllineitems}
\phantomsection\label{\detokenize{documentation:faerun.Faerun}}\pysiglinewithargsret{\sphinxbfcode{\sphinxupquote{class }}\sphinxcode{\sphinxupquote{faerun.}}\sphinxbfcode{\sphinxupquote{Faerun}}}{\emph{title: str = 'python-faerun'}, \emph{clear\_color: str = '\#111111'}, \emph{coords: bool = True}, \emph{coords\_color: str = '\#888888'}, \emph{coords\_box: bool = False}, \emph{view: str = 'free'}, \emph{scale: float = 750.0}}{}
Creates a faerun object which is an empty plotting surface where
layers such as scatter plots can be added.

Constructor for Faerun.
\begin{quote}\begin{description}
\item[{Keyword Arguments}] \leavevmode\begin{itemize}
\item {} 
\sphinxstyleliteralstrong{\sphinxupquote{title}} (\sphinxcode{\sphinxupquote{str}}, optional) \textendash{} The title of the generated HTML file

\item {} 
\sphinxstyleliteralstrong{\sphinxupquote{clear\_color}} (\sphinxcode{\sphinxupquote{str}}, optional) \textendash{} The background color of the plot

\item {} 
\sphinxstyleliteralstrong{\sphinxupquote{coords}} (\sphinxcode{\sphinxupquote{bool}}, optional) \textendash{} Show the coordinate axes in the plot

\item {} 
\sphinxstyleliteralstrong{\sphinxupquote{coords\_color}} (\sphinxcode{\sphinxupquote{str}}, optional) \textendash{} The color of the coordinate axes

\item {} 
\sphinxstyleliteralstrong{\sphinxupquote{coords\_box}} (\sphinxcode{\sphinxupquote{bool}}, optional) \textendash{} Show a box around the coordinate axes

\item {} 
\sphinxstyleliteralstrong{\sphinxupquote{view}} (\sphinxcode{\sphinxupquote{str}}, optional) \textendash{} The view (front, back, top, bottom, left, right, free)

\item {} 
\sphinxstyleliteralstrong{\sphinxupquote{scale}} (\sphinxcode{\sphinxupquote{float}}, optional) \textendash{} To what size to scale the coordinates (which are normalized)

\end{itemize}

\end{description}\end{quote}
\index{\_\_init\_\_() (faerun.Faerun method)@\spxentry{\_\_init\_\_()}\spxextra{faerun.Faerun method}}

\begin{fulllineitems}
\phantomsection\label{\detokenize{documentation:faerun.Faerun.__init__}}\pysiglinewithargsret{\sphinxbfcode{\sphinxupquote{\_\_init\_\_}}}{\emph{title: str = 'python-faerun'}, \emph{clear\_color: str = '\#111111'}, \emph{coords: bool = True}, \emph{coords\_color: str = '\#888888'}, \emph{coords\_box: bool = False}, \emph{view: str = 'free'}, \emph{scale: float = 750.0}}{}
Constructor for Faerun.
\begin{quote}\begin{description}
\item[{Keyword Arguments}] \leavevmode\begin{itemize}
\item {} 
\sphinxstyleliteralstrong{\sphinxupquote{title}} (\sphinxcode{\sphinxupquote{str}}, optional) \textendash{} The title of the generated HTML file

\item {} 
\sphinxstyleliteralstrong{\sphinxupquote{clear\_color}} (\sphinxcode{\sphinxupquote{str}}, optional) \textendash{} The background color of the plot

\item {} 
\sphinxstyleliteralstrong{\sphinxupquote{coords}} (\sphinxcode{\sphinxupquote{bool}}, optional) \textendash{} Show the coordinate axes in the plot

\item {} 
\sphinxstyleliteralstrong{\sphinxupquote{coords\_color}} (\sphinxcode{\sphinxupquote{str}}, optional) \textendash{} The color of the coordinate axes

\item {} 
\sphinxstyleliteralstrong{\sphinxupquote{coords\_box}} (\sphinxcode{\sphinxupquote{bool}}, optional) \textendash{} Show a box around the coordinate axes

\item {} 
\sphinxstyleliteralstrong{\sphinxupquote{view}} (\sphinxcode{\sphinxupquote{str}}, optional) \textendash{} The view (front, back, top, bottom, left, right, free)

\item {} 
\sphinxstyleliteralstrong{\sphinxupquote{scale}} (\sphinxcode{\sphinxupquote{float}}, optional) \textendash{} To what size to scale the coordinates (which are normalized)

\end{itemize}

\end{description}\end{quote}

\end{fulllineitems}

\index{add\_scatter() (faerun.Faerun method)@\spxentry{add\_scatter()}\spxextra{faerun.Faerun method}}

\begin{fulllineitems}
\phantomsection\label{\detokenize{documentation:faerun.Faerun.add_scatter}}\pysiglinewithargsret{\sphinxbfcode{\sphinxupquote{add\_scatter}}}{\emph{name: str, data: Union{[}dict, pandas.core.frame.DataFrame{]}, mapping: dict = \{'c': 'c', 'cs': 'cs', 'labels': 'labels', 's': 's', 'x': 'x', 'y': 'y', 'z': 'z'\}, colormap: Union{[}str, matplotlib.colors.Colormap{]} = 'plasma', shader: str = 'sphere', point\_scale: float = 1.0, max\_point\_size: float = 100.0, fog\_intensity: float = 0.0, saturation\_limit: float = 0.2, categorical: bool = False, interactive: bool = True, has\_legend: bool = False, legend\_title: str = None, legend\_labels: dict = None}}{}
Add a scatter layer to the plot.
\begin{quote}\begin{description}
\item[{Parameters}] \leavevmode\begin{itemize}
\item {} 
\sphinxstyleliteralstrong{\sphinxupquote{name}} (\sphinxcode{\sphinxupquote{str}}) \textendash{} The name of the layer

\item {} 
\sphinxstyleliteralstrong{\sphinxupquote{data}} (\sphinxcode{\sphinxupquote{dict}} or \sphinxcode{\sphinxupquote{DataFrame}}) \textendash{} A Python dict or Pandas DataFrame containing the data

\end{itemize}

\item[{Keyword Arguments}] \leavevmode\begin{itemize}
\item {} 
\sphinxstyleliteralstrong{\sphinxupquote{mapping}} (\sphinxcode{\sphinxupquote{dict}}, optional) \textendash{} The keys which contain the data in the input dict or the column names in the pandas \sphinxcode{\sphinxupquote{DataFrame}}

\item {} 
\sphinxstyleliteralstrong{\sphinxupquote{colormap}} (\sphinxcode{\sphinxupquote{str}} or \sphinxcode{\sphinxupquote{Colormap}}, optional) \textendash{} The name of the colormap (can also be a matplotlib Colormap object)

\item {} 
\sphinxstyleliteralstrong{\sphinxupquote{shader}} (\sphinxcode{\sphinxupquote{str}}, optional) \textendash{} The name of the shader to use for the data point visualization

\item {} 
\sphinxstyleliteralstrong{\sphinxupquote{point\_scale}} (\sphinxcode{\sphinxupquote{float}}, optional) \textendash{} The relative size of the data points

\item {} 
\sphinxstyleliteralstrong{\sphinxupquote{max\_point\_size}} (\sphinxcode{\sphinxupquote{int}}, optional) \textendash{} \textendash{} The maximum size of the data points when zooming in

\item {} 
\sphinxstyleliteralstrong{\sphinxupquote{fog\_intensity}} (\sphinxcode{\sphinxupquote{float}}, optional) \textendash{} \textendash{} The intensity of the distance fog

\item {} 
\sphinxstyleliteralstrong{\sphinxupquote{saturation\_limit}} (\sphinxcode{\sphinxupquote{float}}, optional) \textendash{} \textendash{} The minimum saturation to avoid “gray soup”

\item {} 
\sphinxstyleliteralstrong{\sphinxupquote{categorical}} (\sphinxcode{\sphinxupquote{bool}}, optional) \textendash{} \textendash{} Whether this scatter layer is categorical

\item {} 
\sphinxstyleliteralstrong{\sphinxupquote{interactive}} (\sphinxcode{\sphinxupquote{bool}}, optional) \textendash{} \textendash{} Whether this scatter layer is interactive

\item {} 
\sphinxstyleliteralstrong{\sphinxupquote{has\_legend}} (\sphinxcode{\sphinxupquote{bool}}, optional) \textendash{} \textendash{} Whether or not to draw a legend

\item {} 
\sphinxstyleliteralstrong{\sphinxupquote{legend\_title}} (\sphinxcode{\sphinxupquote{str}}, optional) \textendash{} \textendash{} The title of the legend

\item {} 
\sphinxstyleliteralstrong{\sphinxupquote{legend\_labels}} (\sphinxcode{\sphinxupquote{dict}}, optional) \textendash{} \textendash{} A dict mapping values to legend labels

\end{itemize}

\end{description}\end{quote}

\end{fulllineitems}

\index{add\_tree() (faerun.Faerun method)@\spxentry{add\_tree()}\spxextra{faerun.Faerun method}}

\begin{fulllineitems}
\phantomsection\label{\detokenize{documentation:faerun.Faerun.add_tree}}\pysiglinewithargsret{\sphinxbfcode{\sphinxupquote{add\_tree}}}{\emph{name: str, data: Union{[}dict, pandas.core.frame.DataFrame{]}, mapping: dict = \{'c': 'c', 'from': 'from', 'to': 'to', 'x': 'x', 'y': 'y', 'z': 'z'\}, color: str = '\#666666', colormap: Union{[}str, matplotlib.colors.Colormap{]} = 'plasma', fog\_intensity: float = 0.0, point\_helper: str = None}}{}
Add a tree layer to the plot.
\begin{quote}\begin{description}
\item[{Parameters}] \leavevmode\begin{itemize}
\item {} 
\sphinxstyleliteralstrong{\sphinxupquote{name}} (\sphinxcode{\sphinxupquote{str}}) \textendash{} The name of the layer

\item {} 
\sphinxstyleliteralstrong{\sphinxupquote{data}} (\sphinxcode{\sphinxupquote{dict}} or \sphinxcode{\sphinxupquote{DataFrame}}) \textendash{} A Python dict or Pandas DataFrame containing the data

\end{itemize}

\item[{Keyword Arguments}] \leavevmode\begin{itemize}
\item {} 
\sphinxstyleliteralstrong{\sphinxupquote{mapping}} (\sphinxcode{\sphinxupquote{dict}}, optional) \textendash{} The keys which contain the data in the input dict or DataFrame

\item {} 
\sphinxstyleliteralstrong{\sphinxupquote{color}} (\sphinxcode{\sphinxupquote{str}}, optional) \textendash{} The default color of the tree

\item {} 
\sphinxstyleliteralstrong{\sphinxupquote{colormap}} (\sphinxcode{\sphinxupquote{str}} or \sphinxcode{\sphinxupquote{Colormap}}, optional) \textendash{} The name of the colormap (can also be a matplotlib Colormap object)

\item {} 
\sphinxstyleliteralstrong{\sphinxupquote{fog\_intensity}} (\sphinxcode{\sphinxupquote{float}}, optional) \textendash{} The intensity of the distance fog

\item {} 
\sphinxstyleliteralstrong{\sphinxupquote{point\_helper}} (\sphinxcode{\sphinxupquote{str}}, optional) \textendash{} The name of the scatter layer to associate with this tree layer (the source of the coordinates)

\end{itemize}

\end{description}\end{quote}

\end{fulllineitems}

\index{create\_data() (faerun.Faerun method)@\spxentry{create\_data()}\spxextra{faerun.Faerun method}}

\begin{fulllineitems}
\phantomsection\label{\detokenize{documentation:faerun.Faerun.create_data}}\pysiglinewithargsret{\sphinxbfcode{\sphinxupquote{create\_data}}}{}{{ $\rightarrow$ str}}
Returns a JavaScript string defining a JavaScript object containing the data.
\begin{quote}\begin{description}
\item[{Returns}] \leavevmode
JavaScript code defining an object containing the data

\item[{Return type}] \leavevmode
\sphinxcode{\sphinxupquote{str}}

\end{description}\end{quote}

\end{fulllineitems}

\index{create\_python\_data() (faerun.Faerun method)@\spxentry{create\_python\_data()}\spxextra{faerun.Faerun method}}

\begin{fulllineitems}
\phantomsection\label{\detokenize{documentation:faerun.Faerun.create_python_data}}\pysiglinewithargsret{\sphinxbfcode{\sphinxupquote{create\_python\_data}}}{}{{ $\rightarrow$ dict}}
Returns a Python dict containing the data
\begin{quote}\begin{description}
\item[{Returns}] \leavevmode
The data defined in this Faerun instance

\item[{Return type}] \leavevmode
\sphinxcode{\sphinxupquote{dict}}

\end{description}\end{quote}

\end{fulllineitems}

\index{discrete\_cmap() (faerun.Faerun static method)@\spxentry{discrete\_cmap()}\spxextra{faerun.Faerun static method}}

\begin{fulllineitems}
\phantomsection\label{\detokenize{documentation:faerun.Faerun.discrete_cmap}}\pysiglinewithargsret{\sphinxbfcode{\sphinxupquote{static }}\sphinxbfcode{\sphinxupquote{discrete\_cmap}}}{\emph{n\_colors: int}, \emph{base\_cmap: str}}{{ $\rightarrow$ matplotlib.colors.Colormap}}
Create an N-bin discrete colormap from the specified input map.
\begin{quote}\begin{description}
\item[{Parameters}] \leavevmode
\sphinxstyleliteralstrong{\sphinxupquote{n\_colors}} (\sphinxcode{\sphinxupquote{int}}) \textendash{} The number of discrete colors to generate

\item[{Keyword Arguments}] \leavevmode
\sphinxstyleliteralstrong{\sphinxupquote{base\_cmap}} (\sphinxcode{\sphinxupquote{str}}) \textendash{} The colormap on which to base the discrete map

\item[{Returns}] \leavevmode
The discrete colormap

\item[{Return type}] \leavevmode
\sphinxcode{\sphinxupquote{Colormap}}

\end{description}\end{quote}

\end{fulllineitems}

\index{get\_min\_max() (faerun.Faerun method)@\spxentry{get\_min\_max()}\spxextra{faerun.Faerun method}}

\begin{fulllineitems}
\phantomsection\label{\detokenize{documentation:faerun.Faerun.get_min_max}}\pysiglinewithargsret{\sphinxbfcode{\sphinxupquote{get\_min\_max}}}{}{{ $\rightarrow$ tuple}}
Get the minimum an maximum coordinates from this plotter instance
\begin{quote}\begin{description}
\item[{Returns}] \leavevmode
The minimum and maximum coordinates

\item[{Return type}] \leavevmode
\sphinxcode{\sphinxupquote{tuple}}

\end{description}\end{quote}

\end{fulllineitems}

\index{plot() (faerun.Faerun method)@\spxentry{plot()}\spxextra{faerun.Faerun method}}

\begin{fulllineitems}
\phantomsection\label{\detokenize{documentation:faerun.Faerun.plot}}\pysiglinewithargsret{\sphinxbfcode{\sphinxupquote{plot}}}{\emph{file\_name: str = 'index'}, \emph{path: str = './'}, \emph{template: str = 'default'}, \emph{legend\_title: str = 'Legend'}, \emph{legend\_orientation: str = 'vertical'}}{}
Plots the data to an HTML / JS file.
\begin{quote}\begin{description}
\item[{Keyword Arguments}] \leavevmode\begin{itemize}
\item {} 
\sphinxstyleliteralstrong{\sphinxupquote{file\_name}} (\sphinxcode{\sphinxupquote{str}}, optional) \textendash{} The name of the HTML / JS file

\item {} 
\sphinxstyleliteralstrong{\sphinxupquote{path}} (\sphinxcode{\sphinxupquote{str}}, optional) \textendash{} The path to which to write the HTML / JS file

\item {} 
\sphinxstyleliteralstrong{\sphinxupquote{template}} (\sphinxcode{\sphinxupquote{str}}, optional) \textendash{} The name of the template to use

\item {} 
\sphinxstyleliteralstrong{\sphinxupquote{legend\_title}} (\sphinxcode{\sphinxupquote{str}}, optional) \textendash{} The legend title

\item {} 
\sphinxstyleliteralstrong{\sphinxupquote{legend\_orientation}} (\sphinxcode{\sphinxupquote{str}}, optional) \textendash{} The orientation of the legend (‘vertical’ or ‘horizontal’)

\end{itemize}

\end{description}\end{quote}

\end{fulllineitems}


\end{fulllineitems}



\subsection{Web}
\label{\detokenize{documentation:web}}\index{host() (in module faerun)@\spxentry{host()}\spxextra{in module faerun}}

\begin{fulllineitems}
\phantomsection\label{\detokenize{documentation:faerun.host}}\pysiglinewithargsret{\sphinxcode{\sphinxupquote{faerun.}}\sphinxbfcode{\sphinxupquote{host}}}{\emph{path: str, label\_type: str = 'smiles', theme: str = 'light', label\_formatter: Callable{[}{[}str, int, str{]}, str{]} = None, link\_formatter: Callable{[}{[}str, int, str{]}, str{]} = None, info: str = None, legend: bool = False, legend\_title: str = 'Legend', view: str = 'front', search\_index: int = 1}}{}
Start a cherrypy server hosting a Faerun visualization.
\begin{quote}\begin{description}
\item[{Parameters}] \leavevmode
\sphinxstyleliteralstrong{\sphinxupquote{path}} (\sphinxcode{\sphinxupquote{str}}) \textendash{} The path to the fearun data file

\item[{Keyword Arguments}] \leavevmode\begin{itemize}
\item {} 
\sphinxstyleliteralstrong{\sphinxupquote{label\_type}} (\sphinxcode{\sphinxupquote{str}}) \textendash{} The type of the labels

\item {} 
\sphinxstyleliteralstrong{\sphinxupquote{theme}} (\sphinxcode{\sphinxupquote{str}}) \textendash{} The theme to use in the front-end

\item {} 
\sphinxstyleliteralstrong{\sphinxupquote{label\_formatter}} (\sphinxcode{\sphinxupquote{Callable{[}{[}str, int, str{]}, str{]}}}) \textendash{} A function used for formatting labels

\item {} 
\sphinxstyleliteralstrong{\sphinxupquote{link\_formatter}} (\sphinxcode{\sphinxupquote{Callable{[}{[}str, int, str{]}, str{]}}}) \textendash{} A function used for formatting links

\item {} 
\sphinxstyleliteralstrong{\sphinxupquote{info}} (\sphinxcode{\sphinxupquote{str}}) \textendash{} A string containing markdown content that is shown as an info in the visualization

\item {} 
\sphinxstyleliteralstrong{\sphinxupquote{legend}} (\sphinxcode{\sphinxupquote{bool}}) \textendash{} Whether or not to show the legend

\item {} 
\sphinxstyleliteralstrong{\sphinxupquote{legend\_title}} (\sphinxcode{\sphinxupquote{str}}) \textendash{} The title of the legend

\item {} 
\sphinxstyleliteralstrong{\sphinxupquote{view}} (\sphinxcode{\sphinxupquote{str}}) \textendash{} The view type (‘front’, ‘back’, ‘top’, ‘bottom’, ‘right’, ‘left’, or ‘free’)

\item {} 
\sphinxstyleliteralstrong{\sphinxupquote{search\_index}} (\sphinxcode{\sphinxupquote{int}}) \textendash{} The index in the label values that is used for searching

\end{itemize}

\end{description}\end{quote}

\end{fulllineitems}




\renewcommand{\indexname}{Index}
\printindex
\end{document}